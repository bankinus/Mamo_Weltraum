\documentclass[a4paper, 10pt]{report}
\usepackage[a4paper, left=2cm, right=3cm, top=2cm, bottom=4cm]{geometry}
\usepackage[onehalfspacing]{setspace}
\usepackage{ngerman}
\usepackage[utf8]{inputenc}
\usepackage{ulem}
\usepackage{amsmath}
\usepackage{amsfonts}
\usepackage{amssymb}
\usepackage{graphicx}
\usepackage{textcomp}
\usepackage{subfigure}
\usepackage{cite}
\usepackage[german]{hyperref}
\usepackage{float}
\usepackage{enumitem}

\usepackage{listings}
\usepackage{color}
 
\definecolor{codegreen}{rgb}{0,0.6,0}
\definecolor{codegray}{rgb}{0.5,0.5,0.5}
\definecolor{codepurple}{rgb}{0.58,0,0.82}
\definecolor{backcolour}{rgb}{0.95,0.95,0.92}

\lstdefinestyle{mystyle}{
    backgroundcolor=\color{backcolour},   
    commentstyle=\color{codegreen},
    keywordstyle=\color{magenta},
    numberstyle=\tiny\color{codegray},
    stringstyle=\color{codepurple},
    basicstyle=\footnotesize,
    breakatwhitespace=false,         
    breaklines=true,                 
    captionpos=b,                    
    keepspaces=true,                 
    numbers=left,                    
    numbersep=5pt,                  
    showspaces=false,                
    showstringspaces=false,
    showtabs=false,                  
    tabsize=2
}

\begin{document}
\begin{titlepage}
\centering
\includegraphics[scale=1]{FAU-nat-logo.png}\par
\vspace{2cm}
{\huge\bfseries Kostengünstig in den Weltraum\par}
\vspace{1cm}
{\Large Michael Banken, Veit Simoneit, Dominik Winkel\par}
\vspace{2cm}
{\Large Mathematische Modellierung WS17/18\par}
\vspace{0.5cm}
{Betreut von Herr Prof. Dr. Kräutle\par}
\vfill
{\Large \today\par}

\end{titlepage}

\tableofcontents


\chapter{Einleitung}


Die vorliegende Arbeit beschäftigt sich mit der Frage ob es kostengünstige Alternativen gibt zu Raketen und Spaceshuttles um in den Weltraum, bzw. eine stabile erdnahe Umlaufbahn zu erreichen.\\
\section{Weltraum}

Weltraum bedeutet unendliche Weiten, tiefes schwarzes Nichts, Sterne und Planeten. Seit Jahrtausenden ist der Mensch fasziniert von den Sternen und was jenseits unseres Planeten liegt. In den letzten 150 Jahren lernte die Menschheit zu fliegen und in den letzten 50 Jahren sogar das Weltall zu bereisen. Forschungsdrang bringt uns dazu immer weiter und immer häufiger nach den Sternen zu greifen.
Mit der Zeit stiegen aber die Kosten diese Forschung zu unterhalten, daher beschäftigen sich Forscher mit der Frage, ob es kostengünstige Alternativen gibt in den Weltraum zu gelangen, außer Rakete und Spaceshuttle.\\
Zunächst muss dazu geklärt werden, was genau Weltraum eigentlich bedeutet. Eine einheitliche Definition gibt es nicht. Die meisten Institutionen definieren den Beginn des Weltraums in einer Höhe von 100 km über dem Meeresspiegel, an der sogenannten Kármán Linie \cite[vgl. 2.b)]{fai_weltraum}. Jenseits dieser Höhe ist die Zentrifugalkraft größer als die aerodynamischen Kräfte, daher wird ab 100 km nicht mehr von Luftfahrt, sondern von Raumfahrt gesprochen. 
In \autoref{Orbits} kann man erkennen welche unterschiedlichen Orbits verschiedene Objekte die vom Menschen in den Weltraum befördert wurden einnehmen. Die ISS kreist in etwa 400 km über dem Meeresspiegel, die meisten Navigationssatelliten kreisen in etwa 20000 km von der Erde entfernt und der geostationäre Orbit ist 35800 km über Normalnull.

\begin{figure}[!hbt]
\centering
\includegraphics[scale=0.11]{orbitvergleiche.png} 
\cite{wiki:orbitals}

\caption{Verschiedene Orbits im Vergleich} 
	\label{Orbits}
	
	
\end{figure}

Um vergleichen zu können, ob wir kostengünstig mit unserer Idee in den Weltraum kommen, wissen wir nun, dass wir mindestens in eine Höhe von 100 km über dem Meeresspiegel gelangen müssen. 


\section{Turm oder Aufzug}
Bei der Recherche zu Alternativen für die klassische Raumfahrt stößt man auf diverse Ideen. Wir stellen kurz die zwei berühmtesten vor und werden dann eine dieser Alternativen im Detail betrachten.
Bereits 1975 hat Edward Pearson in seinem Paper ''The orbital tower: a spacecraft launcher using the Earth's rotational energy'' dargelegt wie es theoretisch möglich sein kann einen Turm ins Weltall zu bauen. Pearson beschreibt das grundlegende Problem einen Turm ins Weltall zu bauen sei es, dass ein so hoher Turm entweder unter seinem eigenen Gewicht zusammenbricht, die Basis so unmöglich größ gebaut werden müsste, dass sie hunderte Kilometer im Durchmesser wäre, oder der Turm würde durch den Einfluss des Mondes in Schwingung geraten und zusammenbrechen \cite[vgl. Introduction]{PE75}.\\
Gelöst werden die Probleme durch das Konzept eines freistehenden Turmes. Freistehend bedeutet in diesem Zusammenhang, dass der Turm durch die Zentrifugalkraft der rotierende Erde und deren Schwerkraft im Gleichgewicht gehalten wird.\\
Der gleiche Lösungsansatz wird für die zweite Idee genutzt, den Aufzug ins Weltall. Für den Aufzug möchte man ein Seil ins Weltall spannen und daran Waren, Maschinen, Menschen entlang transportieren. 
Wir haben uns entschlossen diese Möglichkeit genauer zu betrachten und legen in den nächsten Kapiteln dar, wie so ein Aufzug aussehen könnte.
\section{Tabelle der Symbole}

\begin{table}[!htb]
\centering
\begin{tabular}{|l|c|c|r|}

\hline
Symbol  & Bedeutung & Wert \\
\hline
$r_E$	& Radius der Erde	& $6370km$	\\
$m_E$	& Masse der Erde	& $5.98*10^24 kg$\\
$r_{geo}$	& Radius des geostationären Orbits	& $42300km$	\\
$\omega$	& Winkelgeschwindigkeit	& $7.27*10^{-7} s^{-1}$	\\
$G$		& Newtons Gravitationskonstante & $6.67*10^{-11}\frac{Nm^2}{kg^2}$\\
$Z$ & Zugspannung im Aufzugkabel; bei Kabel mit Tapering konstant & \\ 
$F_g$ & Schwerkraft der Erde & $\frac{Gm_E}{r^2}$\\
$F_Z$ & Zentrifugalkraft entstehend aus der Rotation der Erde & \\
$A$ & Durchmesser des Kabels; ohne Tapering als konstant angenommen &\\
$\varrho$ & Dichte des Kabelmaterials &\\
$g_E$ & Erdbeschleunigung auf der Erde & $g_E=\frac{Gm_E}{r_E^2} \sim 9,80665 m/s^2$\\
$\sigma$ & Zugfestigkeit & \\
$s$ & Sicherheitsfaktor, falls $> 1$ &\\
$c$ & verschiedene Konstanten; im Text näher erläutert &\\


%NACHTRAGEN

\hline
\end{tabular}
\caption{Zusammenfassung der wichtigsten Symbole aus dem Text und ihrer Werte, bzw. der Formeln, die sie definieren} \label{kuerzeltab}
\end{table}


\chapter{Aufzug}

Die Grundidee ist es ein Seil zwischen Erde und Weltall zu spannen und entlang dieses Seils mithilfe von sogenannten ''Climbern'', quasi Fahrstuhlkabinen, Maschinen und Menschen ins Weltall zu befördern.

\section{Seil}
Zuerst stellt sich die Frage, ob es möglich ist ein Seil in den Weltraum zu spannen. Vergleichbar mit dem Turm besteht die Problematik darin, dass ein hängendes Seil unter dem Eigengewicht der eigenen Länge irgendwann reißt, oder es müsste sehr dick sein, was es wiederum durch das hohe Eigengewicht zum reißen bringen würde. Die Lösung liegt darin, wie der Tower bei Pearson, dass das Seil mithilfe der Zentrifugalkraft der Erde in sich stabil steht. Wir nehmen folglich an, dass das Seil irgendwo entlang des Äquators auf die Erde reicht und weit ins Weltall hinaus.\\
Betrachten wir die Kräfte die auf ein Seilstück wirken. 

\begin{figure}[h]
	\centering
	\includegraphics[scale=0.3]{seilkraefte.jpg} 
	\caption{Kräfte die auf ein Seilstück der Länge $dr$ wirken}
	\label{Seilkraefte}

\end{figure}


Die meisten Symbole aus dem Text und ihren Wert, bzw. die Formel zur Berechnung findet man in \autoref{kuerzeltab}. In \ref{Seilkraefte} sieht man, zum einen wirkt die Zentrifugalkraft $F_Z$ nach oben, zum anderen wird das Seilstück durch die Gewichtskraft $F_g$ nach unten gezogen. Ferner wirkt auf jedes Teilstück eine Kraft $F_o$ nach oben, die durch die darüberliegenden Teilstücke des Seils ausgeübt wird und analog die Kraft $F_u$ die nach unten wirkt. Die Vektorsumme dieser vier Kräfte muss gleich null sein damit das Element auf das sie wirken im Gleichgewicht ist. Für ein Objekt das sich in geostationärem Orbit befindet, gilt, dass die Zentrifugalkraft gleich der Gravitationskraft ist. Daraus folgt, dass an den Enden des Seils die nach oben wirkende Kraft gleich der nach unten wirkenden Kraft sein muss. Betrachtet man ein Teilstück unterhalb des geostationären Orbits, so ist $F_g > F_Z$ und es muss gelten $F_o > F_u$ damit das Gleichgewicht erhalten bleibt, analog gilt für ein Teilstück oberhalb des geostationären Orbits, dass $F_Z > F_g$ und somit $F_u > F_o$. Aus diesen beiden Feststellungen folgt, dass die Zugspannung eine Funktion ist die mit zunehmender Höhe jenseits des geostationären Orbits abnimmt. Eine freistehende Seilkonstruktion ist eine für die gilt, dass die Zugspannung an beiden Enden Null wird und somit keine Befestigung an einem der beiden Enden braucht um an Ort und Stelle zu bleiben. Das Gesamtbild einer Funktion der Zugspannung ist es, dass sie von Null auf Erdradius Niveau ansteigt bis zu einem Maximalwert am geostationären Orbit und dann wieder abfällt bis sie Null erreicht am Weltraum Ende des Seils (siehe \autoref{fig:Tapering}).\cite[vgl. II Height of a free standing tower at the earth's equator]{AR07} \\

Seien nun $m_E$ die Masse der Erde, $r_E$ der Radius der Erde am Äquator und $\omega$ die Winkelgeschwindigkeit der Erde. Der Radius des geostationären Orbits $r_{geo}=(\frac{Gm_E}{\omega^2})^\frac{1}{3}$.
%NACHTRAGEN - Ist das korrekt/nachvollziehbar für den geostat. Orbit!?
Das Modell, das wir betrachten ist ein freistehendes Kabel mit konstanter Dichte $\varrho$ und konstantem Durchmesser $A$ an allen Stellen des Seils. Aus diesem Kabel betrachten wir einen kleinen Abschnitt $dr$ wie in \autoref{Seilkraefte}, dessen unteres Ende im Abstand $r$ vom Erdmittelpunkt sich befindet. Wie bereits erläutert müssen sich die Kräfte in jedem Abschnitt des Seils aufheben, woraus folgt, dass $F_o + F_Z - F_u - F_g = 0$ gelten muss. Der Teil $F_o - F_u$ kann umgeschrieben werden als $AdZ$, da dies die beiden Kräfte sind die auf die Querschnittsfläche eines Kabelstücks wirken. Dann formen wir die Gleichgewichtsbedingung um in
\begin{align*}
AdZ = F_g - F_Z && \text{mit}	& F_g = \frac{Gm_E(Adr\varrho}{r^2}
\\ 
AdZ = \frac{Gm_E(Adr\varrho)}{r^2} - F_Z		&& \text{mit}	& F_Z = (Adr\varrho)\omega^2r
\end{align*}

folgt
\begin{equation}
AdZ = \frac{Gm_E(Adr\varrho)}{r^2} - (Adr\varrho)\omega^2r
\end{equation}
\label{eq1}
Teilt man nun \autoref{eq1} auf beiden Seiten durch $Adr$, so erhält man die Differentialgleichung
\begin{equation}
\frac{dZ}{dr} = Gm_E\varrho[\frac{1}{r^2} - \frac{r}{r_{geo}^3}]
\end{equation}
\label{eq2}
Multiplizieren wir beide Seiten mit $dr$ 
\begin{equation*}
dZ = (Gm_E\varrho[\frac{1}{r^2} - \frac{r}{r_{geo}^3}])dr
\end{equation*}
Integriert man auf beiden Seiten von $r=r_E$ nach $r=r_{geo}$ 
\begin{align*}
\int_{r_E}^{r_{geo}}1dZ = \int_{r_E}^{r_{geo}}(Gm_E\varrho[\frac{1}{r^2} - \frac{r}{r_{geo}^3}])dr\\
Z(r_{geo}) - Z(r_E) = Gm_E\varrho[\frac{1}{r_E} - \frac{3}{2r_{geo}} + \frac{r_E^2}{2r_{geo}^3}]
\end{align*}
und mit der Anfangsbedingung $Z(r_E)=0$ gilt für die Zugspannung am geostationären Orbit
\begin{equation}
Z(r_{geo}) = Gm_E\varrho[\frac{1}{r_E} - \frac{3}{2r_{geo}} + \frac{r_E^2}{2r_{geo}^3}]
\end{equation}
\label{eq3}

Sei nun $H$ die Entfernung vom Weltraum Ende des Seils bis zum Erdmittelpunkt. Ähnlich zum Schritt von \autoref{eq2} zu \autoref{eq3} integrieren wir \autoref{eq2} von $r=r_{geo}$ bis $r=H$ mit der Bedingung, dass $Z(H)=0$, was einer Zugspannung von Null am Weltraum Ende des Seils entspricht.
Damit ergibt sich
\begin{equation}
Z(r_{geo}) = Gm_E\varrho[\frac{1}{H} - \frac{3}{2r_{geo}} + \frac{H^2}{2r_{geo}^3}]
\end{equation}
\label{eq4}
Jetzt setzen wir die rechten Seiten von \autoref{eq3}und \autoref{eq4} gleich
\begin{equation*}
Gm_E\varrho[\frac{1}{r_E} - \frac{3}{2r_{geo}} + \frac{r_E^2}{2r_{geo}^3}]=Gm_E\varrho[\frac{1}{H} - \frac{3}{2r_{geo}} + \frac{H^2}{2r_{geo}^3}]
\end{equation*}
Da wir wissen, dass $H=r_E$ eine Lösung dieser Gleichung sein muss, können wir das ganze auf die quadratische Gleichung
\begin{equation}
r_EH^2 + r_E^2H-2r_{geo}^3 =0
\end{equation}
\label{eq5}
einstampfen. Die einzige positive Lösung für $H$ ist\\
\begin{equation*}
H = \frac{r_E}{2}[\sqrt{1+8(\frac{r_{geo}}{r_E})^3}-1] = 150000 km
\end{equation*}

Somit ist gezeigt, dass ein stabiles freistehendes Kabel möglich ist und ungefähr $H-r_E \sim 144000 km$ lang sein muss.\cite[vgl. II Height of a free standing tower at the earth's equator]{AR07}\\


Nachdem gezeigt worden ist, dass ein Seil ins Weltall stabil gespannt werden kann, widmen wir uns zwei entscheidenden Designfragen: 
\begin{itemize}
\item Welches Material benötigen wir für das Seil?
\item Wie können wir Stabilität und Kosteneinsparungen vereinen?
\end{itemize}

\section{Materialfrage}
Der limitierende Faktor beim Bau eines Aufzugs oder eines Turms bis zum geostationären Orbit ist das verwendete Material. Das Gewicht des Materials ist ein wichtiger Faktor, jedoch noch elementarer ist die spezifische Höhe des Materials. Spezifische Höhe, auch ''self support length'' genannt ist die maximale Länge eines Pfeilers aus einem spezfischen Material, der nur an der Spitze befestigt ist und der sein Eigengewicht tragen kann.\cite{wiki:Specific_strength}

Ohne Beschränkung der Allgemeinheit nehmen wir für die spezifische Höhe entlang des Seils die gleiche Gravitation von $g_E = 9,80665 m/s^2$ an.
Betrachten wir im Vergleich drei Materialien: Stahl, Kevlar und Carbon Nano Tubes. \cite[vergleiche]{ED00}
\subsection{Stahl}
Das klassische Baumaterial um Wolkenkratzer überall auf der Welt zu bauen: Stahl. Da Stahl per se kein homogenes Material sein muss, betrachten wir die physikalischen Werte für stainless steel ohne die allgemeine Gültigkeit zu verlieren. Die Dichte von Stahl wird angegeben mit $\varrho_S = 7850 kg/m^3$ \cite[Vgl.]{PE75}. Die Zugfestigkeit von Stahl beträgt $2*10^9 kg/m*s^2$ , was $2GPa$ entspricht. Man berechnet die spezifische Höhe nun durch die Formel 
\begin{equation}
\frac{Zugfestigkeit_{Stahl}}{Dichte_{Stahl}*g_E} = \frac{2*10^9}{7850*9,80665}
<=> 25 980 m \sim 26 km 
\end{equation}

Überprüfen wir kurz die Einheiten
\begin{equation}
\frac{\frac{kg}{m*s^2}}{\frac{kg}{m^3}*\frac{m^2}{s^2}}= \frac{\frac{kg}{m*s^2}}{\frac{kg*m^2}{m^3*s^2}} = \frac{1}{\frac{1}{m}} = m
\end{equation}
Bei einer Seildicke von $2 cm$ im Durchmesser und einer Länge von $144.000 km$ bedeutet dies eine Gesamtmasse von $\sim 355.000 t$.\cite{PE75}
Betrachten wir ein anderes Vergleichsmaterial.
\subsection{Kevlar}
Kevlar wird in modernen schusssicheren Westen verwendet, da es leichter als Stahl aber sehr stabil ist. Die Dichte von Kevlar wird mit $\varrho_K = 1440 kg/m^3$ angegeben und liegt somit bei ungefähr einem Sechstel der Dichte von Stahl. Die Zugfestigkeit von Kevlar liegt bei $3,6 GPa$ und ist folglich 1,8-mal zugfester als Stahl. Hieraus folgt die spezifische Höhe für Kevlar mit
\begin{equation}
\frac{Zugfestigkeit_{Kevlar}}{Dichte_{Kevlar}*g_E} = \frac{3,6*10^9}{1440*9,80665}
<=> 254 929 m \sim 254 km 
\end{equation}
Kevlar hat eine spezifische Höhe, die 10-mal so hoch ist wie die von Stahl, jedoch ist es bis jetzt nicht ausreichend um einen Space Elevator zu bauen. \cite{PE75}
\subsection{Carbon Nano Tubes}
\label{sec:cnt}
Die Lösung der Materialfrage liegt in den \textsl{Carbon Nano Tubes}.
Bei einer theoretischen Zugfestigkeit von $30 - 130 GPa$ und einer Dichte von $\varrho_{CNT} = 1300 kg/m^3$ liegt die spezifische Höhe zwischen $2.350 - 10.000 km$. Laut Pearson ist der Bau eines Weltraumaufzugs, bzw. Turms vergleichbar mit dem Bau eines Turms von $4900 km$ in einem gleichmäßigen 1-g-Schwerefeld \cite{PE75}. Und genau diese Voraussetzung erfüllen die Carbon Nano Tubes.

\section{Tapering}
\subsection{Tapering-Modellierung}
Es bleiben zwei Probleme bei unseren Materialanfordungen an unser Seil, die mit unserem bisherigen Modellansatz nur schwer in den Griff zu bekommen sind.

Zum einen sind Carbon Nano Tubes teuer. Selbst vergleichsweise billige Proben sind ab \$60 pro Gram erhältlich. Für Material von hoher Qualität, wie sie für diesen Anwendungszweck erforderlich wären, fallen sogar noch extremere Kosten von \$750 pro Gramm\cite{baughman2002carbon}.

Sogar wesentlich gravierender ist die Gefahr, dass die in Abschnitt \ref{sec:cnt} genannte Spitzenlast von 130 GPa gar nicht zu erreichen sind. Es gibt Gründe anzunehmen, dass Carbon Nano Tubes einer theoretischen Belastbarkeitsgrenze von 45 GPa unterliegen\cite{pugno2007space}.

Beide Bedenken liefern uns gute Gründe so sparsam wie möglich mit unserem Kabelmaterial umzugehen. Da wir bestimmte Mindestanforderungen an die Länge unseres Aufzugs haben, nämlich dass er auf einer Seite bis zur Erdoberfläche reichen sollte und andererseits mindestens bis über den geostationären Orbit hinaus, bleibt uns eigentlich nur die Option an der Fläche zu sparen.\\
Im Falle der Kosten ist offensichtlich, dass eine Reduktion der Gesamtmenge des Materials zu einer Kostenersparnis führt.\\
Im Falle der Belastbarkeit ist dies weniger offensichtlich, da sowohl die maximal mögliche Belastung des Seils, wie auch die auf das Seil wirkenden Kräfte, linear mit der Fläche skalieren. Der Trick ist, dass unser bisheriges Modell von einer konstanten Querschnittfläche A des Seils ausgeht. Ziel bei der Materialfindung war es demnach ein Material mit Zugfestigkeit $\sigma$ und Dichte $\varrho$ zu finden, so dass die Ungleichung \ref{eq:notaper} erfüllt ist.

\begin{equation}
A \cdot \sigma \geq \int_{r_E}^{r} A \cdot \varrho \cdot G \cdot m_E \cdot \frac{1}{\tilde{r}^2} - A \cdot \varrho \cdot \omega^2 \cdot \tilde{r}\ d\tilde{r}
\label{eq:notaper}
\end{equation}
Hierbei ist $A$ die Fläche des Seils, $\sigma$ ist die Zugfestigkeit des Seils und $\varrho$ ist die Dichte.\\
$m_E$ ist die Masse der Erde, $r_E$ der Radius derselbigen, $\omega$ ist die Winkelgeschwindigkeit mit der sich die Erde (und somit auch der Aufzug) dreht.\\
$G$ ist die Gravitationskonstante.

Beachtet man jedoch, dass an unterschiedlichen Stellen im Seil eine unterschiedlich hohe Zugkraft herrscht, so kann man die Materialmenge der entsprechenden Zugkraft anpassen.
Wir variieren A nach der Höhe r und setzen $Z(r) = \sigma \cdot A(r)$ und erhalten die Integralgleichung \ref{eq:taperint}.

\begin{equation}
\sigma \cdot A(r) = \int_{r_E}^{r} A(r) \cdot \varrho \cdot G \cdot m_E \cdot \frac{1}{\tilde{r}^2} - A(r) \cdot \varrho \cdot \omega^2 \cdot \tilde{r}\ d\tilde{r}
\label{eq:taperint}
\end{equation}
Um uns zu vergewissern, dass dieses Integral auch tatsächlich der Zugkraft auf einer Seil-Höhe r entspricht, teilen wir zunächst das Seil in kleinere Seilabschnitte wie in Abbildung \ref{fig:differential} zu sehen.\\
Hierbei entspricht $r$ der Höhe auf der sich der Seilabschnitt befindet.\\
$dr$ entspricht der Höhe des Seilabschnittes selbst, die wir später infinitesimal klein werden lassen.\\
A($r$), bzw. A($r+dr$) entsprechen der Querschnittfläche am unteren und oberen Abschnitt des Seils.\\
$F_g$ entspricht der Gravitationskraft, die den Seilabschnitt nach unten zieht.\\
$F_Z$ entspricht der Zentrifugalkraft, die den Seilabschnitt nach oben zieht.\\
$F_o$, bzw. $F_u$ entsprechen jeweils den Kräften die sich vom oberen, bzw. unteren Rest des Seils auf den Abschnitt auswirken. Sie entsprechen der Summe aller Kräfte die sich auf ihre jeweiligen Abschnitte auswirken.
\begin{figure}[!htb]
	\centering
	\includegraphics{differential}
	\label{fig:differential}
	\caption{Seilabschnitt mit wirkenden Kräften}
\end{figure}
Um nun die Zugkraft in einer Höhe $r$ zu bestimmen, lassen wir nun die Höhe unserer Seilabschnitte $dr$ gegen 0 gehen.\\
Für den nächsten Schritt müssen wir voraussetzen, dass A stetig ist. Wir können dies Problemlos tun, indem wir A einfach auf solche Lösungen Einschränken, die stetig sind, da unser Lösungsweg so oder so genau solche A finden wird. Aus der Stetigkeit von $A$ können wir nun für $dr \rightarrow 0$ folgern, dass $A(r)=A(r+dr)$. Somit ergibt sich für die Masse $m_{dr}$ unserer Seilabschnitte die Gleichung \ref{eq:samass}.

\begin{equation}
m_{dr} = A(r) \cdot \varrho \cdot dr
\label{eq:samass}
\end{equation}
Mit dieser Masse können wir nun auch Gleichungen für die Kräfte $F_g$ (\ref{eq:Fg}) und $F_z$(\ref{eq:Fz}) aufstellen:

\begin{align}
F_g(r) &= A(r) \cdot \varrho \cdot G \cdot m_E \cdot \frac{1}{r^2} \cdot dr\label{eq:Fg}\\
F_z(r) &= A(r) \cdot \varrho \cdot r \cdot \omega^2 \cdot dr\label{eq:Fz}
\end{align}
Wir vernachlässigen hierbei den Unterschied zwischen der Höhe $r$ und der Höhe $r+dr$, da $r+dr\rightarrow r$ für $dr \rightarrow 0$.

Die nötige Zugkraft auf der Höhe $r$ ist nun die Summe aller Differenzen der Kräfte $F_g$ und $F_z$, die sich für jeden (infinitesimal kleinen) Seilabschnitt unterhalb der Höhe $r$ ergeben. Diese Summe entspricht genau dem Integral aus Gleichung \ref{eq:taperint}. Die linke Seite der Gleichung ergibt sich sofort aus der Ersetzung von $Z(r)$ durch $\sigma \cdot A(r)$

Ein großer Vorteil dieser Modellierung gegenüber der Formulierung des Problems mit konstanter Querschnittfläche ist, dass sofern unabhängig von den vorkommenden Konstanten Lösungen für diese Integralgleichung existieren, sich ein stabiles Seil für jedes erdenkliche Material konstruieren lässt, wenn auch nur theoretisch. Es sollte offensichtlich sein, dass sich aus der bloßen theoretischen Existenz dieser Lösung noch keine praktische Umsetzbarkeit ergibt.

\subsection{Lösung für Tapering-Modell}

Um die Lösungen der Integralgleichung zu bestimmen, wenden wir zunächst den Fundamentalsatz der Analysis an und formen die Gleichung so zur Differentialgleichung \ref{eq:taperdgl} um.
\begin{align}
A'(r) \cdot \sigma = A(r) \cdot \varrho \cdot G \cdot m_E \cdot \frac{1}{r^2} - A(r) \cdot \varrho \cdot \omega^2 \cdot r \nonumber\\
A'(r) = \frac{A(r)}{\sigma} (\cdot \varrho \cdot G \cdot m_E \cdot \frac{1}{r^2} - \varrho \cdot \omega^2 \cdot r)
\label{eq:taperdgl}
\end{align}
Die sich ergebende Differentialgleichung besitzt mehrere wichtige Eigenschaften:
\begin{itemize}
[align=left]
\item[gewöhnlich:] Es gibt nur eine Variable, von der die gesuchte Funktion und ihre Ableitungen abhängen $\Rightarrow$ DGL ist gewöhnlich
\item[linear:] Die Funktion und ihre Ableitungen kommen jeweils nur in einfacher Potenz vor $\Rightarrow$ DGL ist linear
\item[homogen:] Aus der Form von Gleichung \ref{eq:taperdgl} ist sofort ersichtlich, dass die Störfunktion einen Wert von 0 hat $\Rightarrow$ DGL ist homogen
\item[Rang 1:] Die höchste vorkommende Ableitung ist die erste Ableitung der gesuchten Funktion. $\Rightarrow$ DGL 1. Ordnung
\item[nicht autonom:] Der Term $\frac{\varrho \cdot G \cdot m_E \cdot \frac{1}{r^2} - \varrho \cdot \omega^2 \cdot r}{\sigma}$ beinhaltet die Variable $r$ von der die gesuchte Funktion abhängt $\Rightarrow$ DGL ist nicht autonom
\item[separierbar:] Separabilität ist sofort aus Form von Gleichung \ref{eq:taperdgl} ersichtlich $\Rightarrow$ DGL ist separierbar
\end{itemize}
Es handelt sich also um eine gewöhnliche homogene lineare Differentialgleichung 1. Ordnung.

In der Literatur\cite{PE75} findet man sie auch in der Form \ref{eq:taperdgllit}.

\begin{equation}
A'(r) \cdot \sigma = \varrho \cdot g_E \cdot {r_E}^2 \cdot (\frac{1}{r^2} - \frac{r}{{r_{geo}}^3}) \cdot A(r)
\label{eq:taperdgllit}
\end{equation}
Auf den ersten Blick sieht dies nach einer völlig anderen Gleichung aus, aber durch einige einfache Umformungen und Ausnutzung der Gleichungen \ref{eq:geosynchron} und \ref{eq:surfaceacc} lässt sich schnell erkennen, dass es sich hierbei um eine equivalente Gleichung handelt.

\begin{align}
\frac{m \cdot m_E \cdot G}{r_{geo}^2} &= m \cdot \omega^2 \cdot r_{geo}\nonumber \\
\omega^2 &= \frac{m_E \cdot G}{r_{geo}^3}
\label{eq:geosynchron}
\end{align}

\begin{align}
g_E &= \frac{m_E \cdot G}{r_E^2} \nonumber \\
G \cdot m_E &= g_E \cdot r_E^2
\label{eq:surfaceacc}
\end{align}

\begin{align*}
A'(r) \cdot \sigma &= A(r) \cdot \varrho \cdot G \cdot m_E \cdot \frac{1}{r^2} - A(r) \cdot \varrho \cdot \omega^2 \cdot r\\
&= \varrho \cdot (G \cdot m_E \cdot \frac{1}{r^2} - \omega^2 \cdot r) \cdot A(r)\\
&=_{\ref{eq:geosynchron}} \varrho \cdot (G \cdot m_E \cdot \frac{1}{r^2} - \frac{m_E \cdot G}{r_{geo}^3} \cdot r) \cdot A(r)\\
&= \varrho \cdot G \cdot m_E \cdot (\frac{1}{r^2} - \frac{r}{r_{geo}^3}) \cdot A(r)\\
&=_{\ref{eq:surfaceacc}} \varrho \cdot g_E \cdot r_E^2 \cdot (\frac{1}{r^2} - \frac{r}{r_{geo}^3}) \cdot A(r)\\
\end{align*}
Dies entspricht exakt der alternativen Gleichung \ref{eq:taperdgllit}. Dennoch werden wir die Rechnung mit der ursprünglichen Gleichung \ref{eq:taperdgl} durchführen. Sowohl die Rechung als auch die Lösung werden sich dementsprechend nicht mit dem decken, was in der Literatur zu finden ist, aber wie gerade gezeigt sind die beiden Gleichungen und damit auch ihre Lösungen equivalent.

Da unsere Differentialgleichung \ref{eq:taperdgl} separierbar ist, können wir sie einfach durch Trennung der Variablen lösen.
\begin{align*}
A'(r) &= \frac{A(r)}{\sigma} \cdot (\frac{\varrho\cdot m_E \cdot G}{r^2} - \varrho \cdot \omega^2 \cdot r)\\
\intertext{Wir beginnen indem wir ein bischen umformen.}
\frac{A'(r)}{\frac{A(r)}{\sigma}} &= (\frac{\varrho\cdot m_E \cdot G}{r^2} - \varrho \cdot \omega^2 \cdot r)\\
A'(r) \cdot \frac{\sigma}{A(r)} &= (\frac{\varrho\cdot m_E \cdot G}{r^2} - \varrho \cdot \omega^2 \cdot r)\\
\intertext{Nun müssen wir beide Seiten integrieren.}
\int_{r_E}^{r} A'(r) \cdot \frac{\sigma}{A(r)} dr &=  \int_{r_E}^{r} \frac{\varrho\cdot m_E \cdot G}{\tilde{r}^2} - \varrho \cdot \omega^2 \cdot \tilde{r} \ d\tilde{r}\\
\intertext{Wir können nun auf der linken Seite die Substitutionsregel anwenden.}
\int_{A(r_E)}^{A(r)} \frac{\sigma}{u} du &=  \int_{r_E}^{r} \frac{\varrho\cdot m_E \cdot G}{\tilde{r}^2} - \varrho \cdot \omega^2 \cdot \tilde{r} \ d\tilde{r}\\
\intertext{Die Stammfunktionen lassen sich nun recht einfach bilden.}
[\sigma \cdot ln(u)]_{A(r_E)}^{A(r)} &=  [\frac{-2\varrho m_E G - \omega^2 r^3}{2 r}]_{r_E}^{r}\\
\sigma ln(A(r)) - \sigma ln(A(r_E)) &= \frac{-2\varrho m_E G - \omega^2 r^3}{2 r} - \frac{-2\varrho m_E G - \omega^2 r_E^3}{2 r_E}\\
ln(A(r)) - ln(A(r_E)) &= \frac{-2\varrho m_E G - \omega^2 r^3}{2 r \sigma} - \frac{-2\varrho m_E G - \omega^2 r_E^3}{2 r_E \sigma}\\
\intertext{Wir können nun jene Teile, die nur von Konstanten abhängen in einem konstanten Faktor zusammenfassen.}
A(r) &= c_I * e^{\frac{-2\varrho \cdot m_E \cdot G - \omega^2 \cdot r^3}{2 r \sigma}}
\end{align*}
Wir erhalten nun also eine Lösung der Form \ref{eq:tapersolution} mit einem konstanten Faktor $c_I$, der beliebig gewählt werden kann, wobei für die Anwendbarkeit nur solche $c_I$ Sinn machen, so dass $c_I>0$.

\begin{align}
A(r) &= c_I * e^{\frac{-2\varrho \cdot m_E \cdot G - \varrho \omega^2 \cdot r^3}{2 r \sigma}}\label{eq:tapersolution}\\
c_I &= A(r_E) \cdot e^{-\frac{-2 c \varrho m_E G - \varrho \omega^2 r_E^3}{2 r_E \sigma}}
\label{eq:taperconstant}
\end{align}
Durch die für das Lösungsverfahren notwendige Integration erhalten wir einen konstanten Faktor $c_I$ der, wie in Gleichung \ref{eq:taperconstant} zu sehen, sich als Produkt eines Anfangswerts für die Kabelfläche auf Höhe der Erdoberfläche und einem konstanten Materialfaktor zusammensetzt. Der Materialfaktor ergibt sich einfach als das inverse des Exponentialausdrucks für die Höhe $r_E$. Auf der Höhe $r_E$ lösen sich dann die beiden Terme gegenseitig auf und es bleibt auf beiden Seiten nur noch $A(r_E)$ stehen.

Analog zur obigen Umformung \ref{eq:taperdgllit} unserer ursprünglichen Differentialgleichung \ref{eq:taperdgl} findet man in der Literatur\cite{PE75} auch die Lösung \ref{eq:tapersolutionlit}.

\begin{equation}
A(r)=A_{geo}*e^\frac{3r_E^2}{2*h*r_{geo}}*e^{\frac{-r_E}{h}*(\frac{r_E}{r}+\frac{r_E*r^2}{2r_{geo}^3})}
\label{eq:tapersolutionlit}
\end{equation}
Basierend auf dieser Lösung lässt sich das so genannte Taperverhältnis bestimmen. Dies ist der Quotient zwischen der Querschnittfläche am am stärksten belasteten und damit dem dicksten Punkt (also dem geosynchronen Orbit) und der Querschnittfläche auf der Erdoberfläche.

\begin{equation}
\frac{A_{geo}}{A_E} = e^\frac{0,776 \cdot r_E}{h}
\label{eq:taperratio}
\end{equation}
$h=\frac{\sigma}{\varrho \cdot g_E}$ mit $\sigma$ ist die als konstant gesetzte Zugfestigkeit. Man erkennt mit zunehmendem $h$ nimmt das Taperverhältnis ab.
Aus Gleichung \ref{eq:taperdgl} folgt, dass die größte Zugkraft am geostationären Orbit auf das Seil wirkt. Dies kann auch daraus geschlussfolgert werden, dass an diesem Punkt zwei gleich große Kräfte in entgegengesetzte Richtungen wirken. Zum einen die Kraft des Seils, das Richtung Erde zieht und zum anderen der Teil des Seils, der vom geostationären Orbit aus von der Erde weg zeigt. Man beachte, dass der geostationäre Orbit genau der Punkt ist, an dem sich die Gravitationskraft der Erde und die Fliehkraft, die auf das Objekt wirkt gegenseitig aufheben, wäre eine Kraft größer, so würde sich das Objekt auf eine andere Umlaufbahn bewegen. Daraus folgt auch, dass die Kraft, die auf den geostationären Punkt wirkt, maximal ist. \\
In Abbildung \ref{fig:Tapering} erkennt man, wie die Dicke des getaperten Kabels bis zum geostationären Orbit zunimmt und danach wieder abnimmt. Die Kurve ist nicht symmetrisch, da die Gravitation zur Erde hin (wo sie am stärksten ist) quadratisch zunimmt, aber die Zentrifugalkraft nach außen hin (wo sie dominiert) nur linear steigt.
\begin{figure}[!htb]
\centering
\includegraphics[scale=0.4]{Tapering.png} 
\caption{Kabel Taper Profil, entnommen aus Edwards(2000)\cite{ED00}}
\label{fig:Tapering}
\end{figure}
\\
Vergleicht man die drei Materialien aus dem vorherigen Abschnitt bezüglich ihres Taper Verhältnisses am geostationären Punkt, so erhält man die folgenden Werte\cite{AR07}:

\begin{itemize}
\item Stahl: \ \ \ \ $\frac{A_{geo}}{A_0} = 1,6 *10^{33}$
\item Kevlar:\ \ \   $\frac{A_{geo}}{A_0} = 2,5 *10^8$
\item CNT: \ \ \ \ $\frac{A_{geo}}{A_0} = 1,6\ bis\ 8,2 $ 
\end{itemize}

Auf Grund der Größenordnung des Verhältnisses ist die einzige Lösung für die Materialwahl Carbon Nano Tubes, Stahl und Kevlar und andere Materialien haben ein zu großes Taperverhältnis.

\subsection{Modell mit Nutzlast}

Unser erweitertes Modell hat uns zwar einen signifikanten Vorteil erbracht, indem es uns erlaubt (zumindest theoretisch) ein Seil zu konstruieren, dass für beliebige feste Materialien stabil ist, aber gleichzeitig macht es eine Annahme, die ein solcherart konstruiertes Seil fundamental als Aufzugkabel unbrauchbar macht. Das Seil hält nämlich genau so viel Belastung aus, dass es sich exakt selbst tragen kann. Um aber einen funktionierenden Aufzug zu konstruieren, brauchen wir ein Seil dass zusätzlich in der Lage ist eine Nutzlast zu tragen.

Glücklicherweise lässt sich unser Modell durch einen einfachen Handkniff erweitern, indem wir in der ursprünglichen Gleichung \ref{eq:taperint} einen Sicherheitsfaktor $s>1$ einführen. Wir multiplizieren nun die rechte Seite der Gleichung auf der sich die auf das Seil wirkenden Kräfte befinden mit diesem Faktor und erhalten die Gleichung \ref{eq:securityint}.

\begin{equation}
\sigma \cdot A_s(r) = s \cdot \int_{r_0}^{r} A_s(\tilde{r}) \cdot (\frac{\varrho\cdot m_E \cdot G}{\tilde{r}^2} - \varrho \cdot \omega^2 \cdot \tilde{r})\ d\tilde{r}
\label{eq:securityint}
\end{equation}
Anstatt dass die maximale Zugkraft, die das Seil aushält genau den wirkenden Kräften entspricht, soll sie nun einem Vielfachem dieser Kräfte entsprechen. Für einen Wert von $s=2$ zum Beispiel, würde das Seil eine Last aushalten, die doppelt so groß wie sein eigenes Gewicht ist. Somit ergibt sich eine effektive Nutzlast, die so groß ist, wie die halbe maximale Zugkraft an der Stelle der geringsten Belastung, also an einem der Endpunkte des Kabels. Aus Sicherheitsgründen sollte man diese Nutzlast natürlich deutlich kleiner wählen.

Auch diese Gleichung kann wieder durch den Fundamentalsatz der Analysis in die Form der Differentialgleichung \ref{eq:securitydgl} gebracht werden.

\begin{equation}
\sigma \cdot A_s'(r) = s \cdot A_s(r) \cdot (\frac{\varrho\cdot m_E \cdot G}{r^2} - \varrho \cdot \omega^2 \cdot r)
\label{eq:securitydgl}
\end{equation}
Wieder handelt es sich also um eine gewöhnliche homogene lineare Differentialgleichung 1. Ordnung. Und insbesondere ist diese Gleichung auch wieder separierbar. Wir können also erneut die Differentialgleichung per Trennung der Variablen lösen. Die Rechnung hierfür sieht völlig analog aus zum vorherigen Modell.

\begin{align*}
A_s'(r) &= \frac{A_s(r)}{\sigma} \cdot s \cdot (\frac{\varrho\cdot m_E \cdot G}{r^2} - \varrho \cdot \omega^2 \cdot r)\\
\frac{A_s'(r)}{\frac{A_s(r)}{\sigma}} &= s \cdot(\frac{\varrho\cdot m_E \cdot G}{r^2} - \varrho \cdot \omega^2 \cdot r)\\
A_s'(r) \cdot \frac{\sigma}{A_s(r)} &= s \cdot (\frac{\varrho\cdot m_E \cdot G}{r^2} - \varrho \cdot \omega^2 \cdot r)\\
\int_{r_E}^{r} A_s'(r) \cdot \frac{\sigma}{A_s(r)} dr &=  \int_{r_E}^{r} s \cdot \frac{\varrho\cdot m_E \cdot G}{\tilde{r}^2} - \varrho \cdot \omega^2 \cdot \tilde{r} \ d\tilde{r}\\
\int_{A_s(r_E)}^{A_s(r)} \frac{\sigma}{u} du &=  s \cdot \int_{r_E}^{r} \frac{\varrho\cdot m_E \cdot G}{\tilde{r}^2} - \varrho \cdot \omega^2 \cdot \tilde{r} \ d\tilde{r}\\
[\sigma \cdot ln(u)]_{A_s(r_E)}^{A_s(r)} &=  s \cdot [\frac{-2\varrho m_E G - \omega^2 r^3}{2 r}]_{r_E}^{r}\\
\sigma ln(A_s(r)) - \sigma ln(A_s(r_E)) &= s \cdot \frac{-2\varrho m_E G - \omega^2 r^3}{2 r} - s \cdot \frac{-2\varrho m_E G - \omega^2 r_E^3}{2 r_E}\\
ln(A_s(r)) - ln(A_s(r_E)) &= s \cdot \frac{-2\varrho m_E G - \omega^2 r^3}{2 r \sigma} - s \cdot \frac{-2\varrho m_E G - \omega^2 r_E^3}{2 r_E \sigma}\\
A_s(r) &= c_I * e^{s \cdot \frac{-2\varrho \cdot m_E \cdot G - \omega^2 \cdot r^3}{2 r \sigma}}
\end{align*}
Die Lösung sieht also aus, wie in Gleichung \ref{eq:securitysolution} zu sehen.

\begin{align}
A_s(r) &= c_{sI} * e^{s \cdot \frac{-2\varrho \cdot m_E \cdot G - \varrho \omega^2 \cdot r^3}{2 r \sigma}}\label{eq:securitysolution}\\
c_{sI} &= A_s(r_E) \cdot e^{- s \cdot \frac{-2 c \varrho m_E G - \varrho \omega^2 r_E^3}{2 r_E \sigma}}
\label{eq:securityconstant}
\end{align}

Wieder haben wir einen Konstanten Faktor $c_{sI}$ der, wie in Gleichung \ref{eq:securityconstant} zu sehen, sich als Produkt eines Anfangswerts für die Kabelfläche auf Höhe der Erdoberfläche und einem konstanten Materialfaktor zusammensetzt. Der Unterschied ist, dass wir dieses mal einen größeren Nutzen daraus ziehen können, dass wir $A_s(r_E)$ beliebig wählen können, da unsere Nutzlast genau $(c-1)\cdot \sigma \cdot A_s(r_E)$ entspricht. Je größer wir also unseren Anfangswert für $A_s(r_E)$ wählen, umso größer wird auch die Nutzlast.

Durch eine kleine Umformung erhalten wir die alternative Darstellung unserer Lösung \ref{eq:securitysolutionalt}

\begin{align}
A_s(r) &= c_{sI} * e^{\frac{-2\varrho \cdot m_E \cdot G - \varrho \omega^2 \cdot r^3}{2 r \frac{\sigma}{s}}}\label{eq:securitysolutionalt}\\
c_{sI} &= A_s(r_E) \cdot e^{- \frac{-2 c \varrho m_E G - \varrho \omega^2 r_E^3}{2 r_E \frac{\sigma}{s}}}
\label{eq:securityconstantalt}
\end{align}
In dieser Form ist sofort ersichtlich, dass es sich bei unserer Lösung für ein getapertes Kabel mit Sicherheitsfaktor s um genau die Lösung handelt, die sich für ein Material dessen Zugfestigkeit dem $\frac{1}{s}$-fachen des eigentlichen Materials entspricht.

Mit dieser Erkenntnis lässt sich auch Gleichung \ref{eq:taperratio} an das neue Modell anpassen in Form von Gleichung \ref{eq:sectaperratio}.

\begin{align}
\frac{A_s(r_{geo})}{A_s(r_E)} &= e^\frac{0,776 \cdot r_E}{\frac{h}{s}} \nonumber\\
\frac{A_s(r_{geo})}{A_s(r_E)} &= e^\frac{s \cdot 0,776 \cdot r_E}{h} \nonumber\\
\frac{A_s(r_{geo})}{A_s(r_E)} &= (e^\frac{0,776 \cdot r_E}{h})^s \nonumber\\
\frac{A_s(r_{geo})}{A_s(r_E)} &= (\frac{A(r_{geo})}{A(r_E)})^s
\label{eq:sectaperratio}
\end{align}
Mit zunehmendem Sicherheitsfaktor s, wächst also das Taperverhältnis exponentiell.

\chapter{Implementierung des Modells}
Im nächsten Schritt wollen wir uns mit der Implementierung des mathematische Modells genauer befassen. Hier wird als erster der Code detailliert vorgestellt, um daraufhin die daraus resultierenden Ergebnisse und Zahlen in Verbindung mit der Realität zu bringen. Dies bildet die Grundlage für das letzte Kapitel, in dem weitere Schwierigkeiten dargestellt werden sollen und ein Ausblick in die Zukunft gegeben wird.
\section{Der zugrundeliegende Code}
Die erste Frage, die sich bei einer Implementierung eines mathematischen Modells stellt, ist die Frage nach einer geeigneten Programmiersprache. Hier fällt die Wahl sehr schnell schon auf Python, welche von Grund auf Bibliotheken wie numpy und scipy mit sich bringt. Durch beide Bibliotheken lassen sich zum Beispiel das Bilden von Integralen und die Berechnungen von Exponentialfunktionen leicht umsetzen. 
Im Folgenden stellen wir nun die verschiedenen Abschnitte der Pythonimplementierung vor.\\
Das Ziel unseres Programms ist die Berechnung verschiedener Massen des Aufzugkabels in Abhängigkeit mehrerer Faktoren wie Länge des Kabels, Schnittfläche auf Höhe des Erdradius oder Größe des Sicherheitsfaktors.
 
\lstset{style=mystyle}
\begin{lstlisting}[language=Python, caption=Python Variableninitialisierung]
import math
import scipy.integrate as integrate
import numpy as np
import matplotlib.pyplot as plt

#Initialisiere Konstanten/Variablen
r_E= np.float128(6370000) # in m
r_geo=np.float128(42300000) # in m
H = np.float128(100000000) # in m

w=np.float128(7.29*10**-5) # in 1/s
G =np.float128(6.67*10**-11) #Gravitationskonstante in Nm^2/kg^2
rho=np.float128(1500) # Dichte
sigma=np.float128(100000000000) #Zugfestigkeit irgendwo zwischen 30 und 130 GPa
m_E=np.float128(5.972*10**24) # in kg
s=np.float128(2) #Sicherheitsfaktor, wenn > 1
A_E=np.float128(1.5*10**-7)
A_mehrere = [0.1*10**-7, 0.2*10**-7, 0.3*10**-7, 0.4*10**-7, 0.5*10**-7, 1*10**-7, 1.5*10**-7, 3*10**-7, 10*10**-7, 20*10**-7]

\end{lstlisting}
Als erster Schritt werden die benötigten Konstanten initialisiert. Hierunter fallen der Erdradius \( r_E \), die Entfernung zum geostationären Orbit \( r_{geo} \), die absolute Länge des Seils \( H \), die Winkelgeschwindigkeit der Erde \(\omega \), die Gravitationskonstante \(\ G \), die Zugfestigkeit des potentiell verwendeten Materials \(\varrho \) und die Dichte des Materials \(\sigma \). Als letzte Konstante wählen wir die Schnittfläche \( A_E \) des Seils auf Höhe des Erdradiuses. Letztere Konstante kann später auch als variable angesehen werden, aufgrund von Vereinfachungen von Berechnungen wird dieser Wert aber als Konstante initialisiert, weswegen für spätere Kalkulationen ein Array mit mehreren Werten \( A_E \) befüllt wird.\
Der in Kapitel 3 eingeführte Sicherheitsfaktor wird bei der ersten Initialisierung auf 2 festgelegt. Wir werden später jedoch noch erkennen, dass dies ein äußerst hoher Wert ist und in der Praxis meist kleinere Werte nahe bei Null als sinnvoll erachtet werden können.

\begin{lstlisting}[language=Python, caption=Aufstellen der Funktion zur Berechnung der Schnittfläche] 
def A(r):
  return A_E*np.exp(-(s(-2*rho*m_E*G/2*r_E+rho*w**2*r_E**2)/(2*r_E*sigma)))*np.exp((s*2(rho*m_E*G/r+rho*w**2*r**3)/(2*sigma*r)))
\end{lstlisting}
Die nächste Programmzeile in unserer Implementierung stellt die Formel 
\begin{equation}
A(r)=A_E \cdot e^\frac{-s(-2\varrho m_E G-\varrho\omega^2r_E^2)}{2r_E\sigma} \cdot e^\frac{s(-2\varrho m_E G-\varrho\omega^2r^3)}{2r\sigma}
\end{equation}
dar, welche zu jeder Eingabe die Schnittfläche des Seils in Höhe \( r \) ausgibt. Dies bildet das Kernstück des Programms, worauf deswegen im weiteren Programm immer wieder Bezug genommen wird. Die Herleitung jener Formel wurde bereits in vorangegangenen Kapitel aufgezeigt. Wie anfangs bereits geschildert interessieren wir uns hauptsächlich für die Masse des vorliegenden Kabels. Um diese zu berechnen, machen wir uns folgende Formel zu Nutze\cite{SPSC}: \\
\begin{equation}
m_{Kabel} = \varrho \cdot \int_{r_E}^{H} A(r) dr
\end{equation}
Um das Programm nicht für jeden möglichen Wert \( A_E \) neu starten zu müssen, wurde wie bereits beschrieben ein Array hierfür angelegt, welches in einer for-Schleife durchlaufen und die dazugehörige Masse berechnet wird. Diese wird in einem noch unbefüllten Array gespeichert, welches zuvor initialisiert wurde. Für die Berechnung des Integrals wird die bereits erwähnte Bibliothek scipy verwendet.

\begin{lstlisting}[language=Python, caption=Hinzufügen von Funkionalität]
weights_rope = []

for a in A_mehrere:
  A_E = a
  m_Kabel= integrate.quad(A, r_E, H)
  weights_rope.append(rho*(m_Kabel1[0]))
  
# Gewicht Seil
print("-----Gewicht Seile-----")
print(weights_rope)
  
\end{lstlisting}
Gibt man nun die Werte aus, die in \( m_{Kabel} \) gespeichert worden sind, so erhält man erste Massenwerte. Hierbei ist die Länge des Seils 100000 km lang, die Zugfestigkeit liegt bei 100 GPa und der Sicherheitsfaktor wurde auf 2 gesetzt.\\
\begin{table}[htb]
\centering
\begin{tabular}{|l|r|}

\hline
\( A_E \) in \textmu \( m^2 \) & Gewicht des Seils in kg\\
\hline
0,1	& 5.008\\
0,2	& 10.016\\
0,3	& 15.024\\
0,4	& 20.032\\
0,5	& 25.040\\
1	& 50.080\\
1,5	& 75.119\\
3	& 150.239\\
10	& 500.796\\
20	& 1.001.591\\
\hline
\end{tabular}
\caption{Schnittflächen und Massen des Seils} \label{tab:sometab}
\end{table}\\
Wie man nun an der Tabelle deutlich erkennen kann, verhalten sich die Schnittfläche und das Gewicht des Seils zum Einen natürlich direkt proportional. Zum Anderen steigt das Gewicht linear an, was auch schon an Formel (5.1) klar sichtbar ist. Der erste Teil \( A_E \) bildet bei Festlegung aller anderen Werte die einzige Variable, was aus der Formel eine einfach zu lösende lineare Gleichung macht.\\
Nun gehen wir einen Schritt weiter und wollen den Nutzen des Seils berechnen, was auf das Maximale Gewicht eines Lifters, der am Seil nach oben gezogen werden kann, hinausläuft. Das effektive Gewicht des Lifters mit der Massen \( m_L \) auf der Erdoberfläche ist \( m_L \cdot (g - \omega^2 \cdot r_E) \) \cite{AR07}. Wegen der Zugspannung im Kabel muss diese Gewicht nun von der nach oben gerichteten Kraft \( A_E \cdot \sigma \) entgegengewirkt werden, was zu der Gleichung\\
\begin{equation}
m_L \cdot (g - \omega^2 \cdot r_E) =  A_E \cdot \sigma
\end{equation}
führt, welche nun nach \( m_L \) aufgelöst werden kann und somit die Gleichung (5.4) zur Folge hat, in welcher der Sicherheitsfaktor \( s \) bereits berücksichtigt wird \\
\begin{equation}
m_L =   \frac{A_E \cdot (\sigma - \frac{\sigma}{s})} {(g - \omega^2 \cdot r_E)}
\end{equation}
\\
Es ist nun besonderes Augenmerk auf die Zugfestigkeit \( \sigma \) in Verbindung mit unserem Sicherheitsfaktor s zu legen. Setzt man den Sicherheitsfaktor auf 1, hat dies zu Folge, dass der Term   \((\sigma - \frac{\sigma}{s})\) gleich 0 wird und die maximale Masse des Lifters ebenfalls auf 0 fällt. Diese Erkenntnis wird ebenfalls von der in Formel (5.1) berechneten Schnittfläche unterstützt. Ein Sicherheitsfaktor von 1 bedeutet nämlich genau dies; ein Seil, welches von der Erdoberfläche knapp 100000 km in das Weltall gespannt wird und sich genau selbst hält.
Die Formel (5.4) wird nun ebenfalls der Implementierung hinzugefügt. Hierfür erweitern wir Listing 5.3 zu folgendem Code:\\
\begin{lstlisting}[language=Python, caption=Hinzufügen der Berechnungen zur Ermittlung des max. Gewichts des Lifters]
weights_rope = []
weights_lifter = []
A_geo = []
A_r_vergleich = []
TaperVergleich = []

for a in A_mehrere:
  A_E = a
  m_Kabel= integrate.quad(A, r_E, H)
  weights_rope.append(rho*(m_Kabel1[0]))
  weights_lifter.append((a*(sigma-sigma/s))/((G*m_E/r_E**2)-w**2*r_E))
  A_geo.append(testrun(R_geo))
  TaperVergleich.append(a*(np.exp(0.776*R_e/r_geo))**1)
  
#Gewicht Seil
print("-----Gewicht Seile-----")
print(weights_rope)

#Gewicht Lifter
print("-----Max Gewicht Lifter-----")
print(weights_lifter)

#A_geo
print("-----A_Geo-----")
print(A_geo)
\end{lstlisting}
Zusätzlich interessiert uns noch das Taperverhältnis und die dazugehörige Schnittfläche \( A_geo \) am geostationären Orbit, welche ebenfalls implementiert worden ist. Das Programm wird nun, mit gleichen Werten wie bereits für Tabelle 5.1 verwendet, ausgeführt. \\


\begin{table}[htb]
\centering
\begin{tabular}{|l|c|c|r|}
\hline

Höhe des Seils in km & Dichte des Seils in \( \frac{kg}{m^3} \)  & Zugfestigkeit \( \sigma \) in GPA & Sicherheitsfaktor \\
\hline
100000 &	1.500 &	100 & 2\\
& & & \\

\hline
\( A_E \) in \textmu \( m^2 \) & Gewicht des Seil in kg & Max. Gewicht des Lifters in kg & \( A_geo \) in \textmu \( m^2 \) \\

\hline
0,1	& 5.008	& 51,10971746	& 0,428629264\\
0,2	& 10.016	& 102,2194349	& 0,857258527\\
0,3	& 15.024	& 153,3291524	& 1,285887791\\
0,4	& 20.032	& 204,4388699	& 1,714517054\\
0,5	& 25.040	& 255,5485873	& 2,143146318\\
1	& 50.080	& 511,0971746	& 4,286292636\\
1,5	& 75.119	& 766,645762	& 6,429438954\\
3	& 150.239	& 1533,291524	& 12,85887791\\
10	& 500.796	& 5110,971746	& 42,86292636\\
20	& 1.001.591	& 10221,94349	& 85,72585272\\

\hline
\end{tabular}
\caption{Daten des zweiten Durchlaufs} \label{tab:sometab}
\end{table}
\section{Die Ergebnisse der Berechnungen}
Wie auch schon im ersten Durchlauf kann man in Tabelle 5.2 die Linearität sowohl der Masse des Seils, als nun aber auch des maximalen Gewicht des Lifters klar sehen. Das Taperverhältnis von knapp 4,3 ist äußerst gut, hierzu sind aber bereits Erläuterungen in der Arbeit vorgestellt worden.\\
In Tabelle 5.3 sind für einen weiteren Durchlauf bis auf die Änderung der Zugfestigkeit keine Änderungen vorgenommen worden. \( \sigma \) ist testweise auf den bisher angenommenen möglichen Maximalwert von Carbon-Nano-Tubes gesetzt worden, welche bei 300 GPa liegt. Dies ist natürlich weit von der Realität entfernt, soll aber nochmals verdeutlichen, welche starken Auswirkungen das Material des Seils auf das Modell und die Realisierbarkeit des Projekts hätte.

\begin{table}[htb]
\centering
\begin{tabular}{|l|c|c|r|}
\hline

Höhe des Seils in km & Dichte des Seils in \( \frac{kg}{m^3} \)  & Zugfestigkeit \( \sigma \) in GPA & Sicherheitsfaktor \\
\hline
100000 &	1.500 &	300 & 2\\
& & & \\

\hline
\( A_E \) in \textmu \( m^2 \) & Gewicht des Seil in kg & Max. Gewicht des Lifters in kg & \( A_geo \) in \textmu \( m^2 \) \\

\hline
0,1	&2.135	&153,3291524&	0,162440359\\
0,2	&4.269	&306,6583048	&0,324880717\\
0,3	&6.404	&459,9874572	&0,487321076\\
0,4	&8.538&	613,3166096	&0,649761434\\
0,5	&10.673&	766,645762&	0,812201793\\
1	&21.346&	1533,291524&	1,624403585\\
1,5	&32.019&	2299,937286&	2,436605378\\
3	&64.037&	4599,874572&	4,873210756\\
10	&213.457&	15332,91524&	1,624403585\\
20	&426.915&	30665,83048	&3,248807171\\

\hline
\end{tabular}
\caption{Daten des dritten Durchlaufs} \label{tab:sometab}
\end{table}

Man kann außerdem erkennen, dass auch unsere Berechnungen das bestmögliche Taperverhältnis von gut 1,6 als Ergebnis bekommen, welches in Kapitel 2 näher erläutert worden ist. Bisher haben wir uns jedoch ausschließlich darauf fokussiert, einzelne Werte im Exponetialteil der Gleichung zu ändern und dabei alle anderen Variablen als Konstanten zu betrachten, was zu der Linearität der Ergebnisse geführt hat. Daher wollen wir nun in der abschließenden Rechnung nicht den linearen Faktor \( A_E \), sondern vielmehr den eingeführten Sicherheitsfaktor \( s \) als Variable ansehen. Der Faktor  \( A_E \) soll nun als Konstante den Wert 1 cm annehmen. Da diese Berechnungen nicht ohne Weiteres mit der aktuellen Implementierung getätigt werden können, muss der Code um Folgendes erweitert werden:

\begin{lstlisting}[language=Python, caption=Berechnungen für konstante Schnittfläche]
#Berechnung Gewichte mit A_E fest
for s in sicherheit:
  sigma=np.float128(100000000000)
  sigma = sigma/s
  m_Kabel= integrate.quad(A, r_E, H)
  weights_rope_dif.append(rho*(m_Kabel[0]))
  weights_lifter_dif.append((A_E*(sigma-sigma/s))/((G*m_E/r_E**2)-w**2*r_E))
  
#Gewicht Seil mit versch Sicherheitsfaktoren
print("-----Gewicht Seile-----")
print(weights_rope_dif)

#Gewicht Lifter mit versch Sicherheitsfaktoren
print("-----Max Gewicht Lifter-----")
print(weights_lifter_dif)
\end{lstlisting}

Hier wird schlichtweg der Sicherheitsfaktor neu angepasst und die berechneten Werte in ein Array gespeichert, welches am Ende ausgegeben wird. Die daraus folgenden Daten in Tabelle 5.4 lasen nun klar erkennen, dass der Sicherheitsfaktor, wie auch angenommen, exponentielle Auswirkungen auf das Gewicht des Seils und logarithmische Auswirkungen auf das maximale Gewicht des Lifters hat.

\begin{table}[htb]
\centering
\begin{tabular}{|l|c|c|r|}
\hline

Höhe des Seils in km & Dichte des Seils in \( \frac{kg}{m^3} \)  & Zugfestigkeit \( \sigma \) in GPA &  \\
\hline
100000 &	1.500 &	100 & \\
& & & \\

\hline
\( A_E \) in \textmu \( m^2 \) & Sicherheitsfaktor \( s \)& Gewicht des Seil in kg & Max. Gewicht des Lifters in kg \\

\hline
1	&1,01	&26539,04125	&10,12073613\\
1	&1,1	&28104,35585	&92,92675903\\
1	&1,2	&29955,237	&170,3657249\\
1	&1,5	&36294,92424	&340,7314498\\
1	&2	&50079,56659	&511,0971746\\
1	&3	&95968,39224	&681,4628995\\
1	&4	&185271,5471	&766,645762\\
1	&5	&359894,4672	&817,7554794\\
1	&10	&10641402,96	&919,9749144\\
1	&20	&11237452274	&971,0846318\\
\hline
\end{tabular}
\caption{Daten des dritten Durchlaufs} \label{tab:sometab}
\end{table}

\begin{figure}[htb]
    \subfigure[exponentielle Auswirkung auf das Gewicht des Seils]{\includegraphics[width=0.49\textwidth]{exponentiell.png}} 
    \subfigure[logarithmische Auswirkung auf das max. Gewicht des Lifters]{\includegraphics[width=0.49\textwidth]{logarithmisch.png}} 
\caption{Auswirkung der Änderungen des Sicherheitsfaktors \( s \)} 
\end{figure} 

Wie man nun vor allem aus Tabelle 5.4 und Abbildung 5.1 herauslesen kann, wir nicht sofort ersichtlich, welcher Sicherheitsfaktor \( s \)  in Kombination mit welcher Schnittfläche \( A_E \) zu wählen ist, um eine bestmögliche Kombination der drei Faktoren zu bekommen. Eine genauere Betrachtung solcher Fragestellungen und der aufgezeigten Ergebnisse soll nun im letzten Abschnitt der Arbeit stattfinden.  

\chapter{Interpretation und Ausblick}
Im nachfolgenden Kapitel werden zwei Dinge weiter beleuchtet. Zum Einen werden die im vorangehenden Abschnitt berechneten Ergebnisse interpretiert und ein Bezug zur Realität dargestellt. Zum Anderen besteht der letzte Teil der Arbeit darin, einen kurzen Ausblick darüber zu geben, was in Zukunft möglich ist, sein kann und was die größten Herausforderungen bei der Konzipierung und der Durchführung des Baus eines Weltraumlifts sind.
\section{Interpretation der Ergebnisse}
Wie bereits kurz angeschnitten, sind exakte Aussagen über die beste Möglichkeit für einen Weltraumlift schwierig zu treffen. Beispielsweise sind wir bisher davon ausgegangen, dass das Seil des Lifts 100.000 km lang sein wird, woraus folgt, dass der Aufzug zusätzlich ein Gegengewicht benötigt, wie in Kapitel 4 aufgezeigt. Es hängt also stark davon ab, wie ein solches Gegengewicht in eine Entfernung von 100.000 km gebracht und am Aufzug befestigt werden kann. Die Masse eines solchen Gegengewichts hängt natürlich wieder direkt mit der Länge des Seils zusammen, wird aber auch von der Dichte eines solchen beeinflusst. Letztere schwankt stark in Abhängigkeit von dem benutzten Material. In unseren Berechnungen haben wir einen erhöhten Wert für die Dichte gewählt, um möglichen Schwankungen in der Beschaffenheit bereits vorzubeugen. Bei stärkeren Materialien ist sogar von einer größeren Dichte auszugehen, was ein schwereres Seil zur Folge hat \cite{DMOS}. Das nächste brisante Thema bei der Konzeption eines Weltraumlifts stellt die Zugfestigkeit des Materials dar. Schon in Kapitel 4 ist gezeigt worden, welche großen Unterschiede hier bei den Materialien auftreten. Schon hier ist klar geworden, dass Carbon-Nano-Tubes die einzige sinnvolle Möglichkeit bieten, den Lift in die Realität umzusetzen. Jedoch besteht sogar innerhalb des Materials in der Wissenschaft keine Einigkeit über die Zugfestigkeit [VEIT GLEICHE ZITATION WIE CNT OBEN], so dass wir sowohl mit einem leicht erhöhten Wert über em Durchschnitt als auch mit einem möglichen Maximalwert für \( \sigma \) gerechnet haben.\\
Möchte man die Daten nun in Bezug zur Realität setzen, kann man dies mit folgenden Überlegungen machen. Wie in \cite{ED00} beschrieben, ist eine sinnvolle Herangehensweise, um das Seil in den Weltraum zu bringen, jenes mit Hilfe von Raketen zum geostationären Orbit zu schicken und von dort aus die Errichtung bzw. das Abseilen zu starten. Nimmt man nun eine Rakete mit einer maximalen Ladung von ca 5.000 kg \cite{ED00}, so kann man nach Fertigstellung des Seiles laut Tabelle 5.2 einen Lifter mit gerade einmal der Masse von 51 kg am Aufzug befestigen. Aus diesem Grund wird nicht nur eine Rakete, sondern mehrere Raketen zur initialen Errichtung des Aufzugs in den Weltraum geschickt. um beispielsweise also eine Traglast von gut 500 kg zu erreichen werden zehn Raketenstarts benötigt. Der Lifter kann nun dazu verwendet werden, das Seil weiterhin zu verstärken und somit über die Zeit immer schwerere Traglasten ermöglichen\cite{ED00}. 
\section{Ausblick}
Vergleichbare Überlegungen wie im letzten Abschnitt mit Werten aus Tabelle 5.3 führen hierauf eine Starttraglast des Lifters von mehr als 3.500 kg. Man kann also klar erkennen, dass die Wahl des geeigneten Materials essentiell ist. Nimmt man den theoretisch möglichen Maximalwert der Carbon-Nano-Tubes ist die Überlegung eines Weltraumlifts plötzlich nicht mehr abwegig. Die Kehrseite hierbei ist jedoch, dass das Projekt jenes Lifts grundsätzlich mit dieser Fragestellung steht und fällt; ein Material mit einer solch hohen Zugfestigkeit ist fundamental für die Errichtung. Doch auch wenn man ein solches Seil als gegeben annimmt erkennt man schnell, dass weitere Probleme auftreten werden, welche jedoch den Rahmen dieser Arbeit sprengen würden:\\
Wie sieht es mit der Sicherheit des Aufzugs aus? Was passiert, wenn dieser doch einmal einstürzt, welche Folgen hätte dies für den Lebensraum unter jenem? Wie wird das Gegengewicht angetrieben, was sich ja offensichtlich hinter dem geostationären Orbit befindet und daher immer hinterher gezogen wird? Bei einer Verkleinerung der Länge des Seils, woher soll ein so großes Gegengewicht kommen, und hat dieses Auswirkungen auf das Kräfteverhältnis im Weltall? Viele dieser Fragen sind bis heute um stritten oder gar ungeklärt. Ein Grund, wieso sich noch vergleichbar wenige mit diesen Themen auseinander setzen könnte sein, dass ein Weltraumlift ohne die Entwicklung bzw. Erforschung eines geeigneten Materials für das Seil sowieso unrealistisch scheint. 

\bibliography{mybib}{}
\bibliographystyle{plain}

\end{document}
