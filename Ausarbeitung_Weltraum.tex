\documentclass[a4paper, 10pt]{report}
\usepackage[a4paper, left=2cm, right=3cm, top=2cm, bottom=4cm]{geometry}
\usepackage{ngerman}
\usepackage[utf8]{inputenc}
\usepackage{ulem}
\usepackage{amsmath}
\usepackage{amsfonts}
\usepackage{amssymb}
\usepackage{graphicx}

\begin{document}
\begin{titlepage}
\centering
\includegraphics[scale=1]{FAU-nat-logo.png}\par
\vspace{2cm}
{\huge\bfseries Kostengünstig in den Weltraum\par}
\vspace{1cm}
{\Large Michael Banken, Veit Simoneit, Dominik Winkel\par}
\vspace{2cm}
{\Large Mathematische Modellierung WS17/18\par}
\vspace{0.5cm}
{Betreut von Herr Prof. Dr. Kräutle\par}
\vfill
{\Large \today\par}

\end{titlepage}

\tableofcontents


\chapter{Einleitung}


Die vorliegende Arbeit beschäftigt sich mit der Frage ob es kostengünstige Alternativen gibt zu Raketen und Spaceshuttles um in den Weltraum, bzw. eine stabile erdnahe Umlaufbahn zu erreichen.\\
\chapter{Weltraum}

Weltraum bedeutet unendliche Weiten, tiefes schwarzes Nichts, Sterne und Planeten. Seit Jahrtausenden ist der Mensch fasziniert von den Sternen und was jenseits unseres Planeten liegt. In den letzten 150 Jahren lernte die Menschheit zu fliegen und in den letzten 50 Jahren sogar das Weltall zu bereisen. Forschungsdrang bringt uns dazu immer weiter und immer häufiger nach den Sternen zu greifen.
Mit der Zeit stiegen aber die Kosten diese Forschung zu unterhalten, daher beschäftigen sich Forscher mit der Frage, ob es kostengünstige Alternativen gibt in den Weltraum zu gelangen, außer Rakete und Spaceshuttle.\\
Zunächst muss dazu geklärt werden, was genau Weltraum eigentlich bedeutet. Eine einheitliche Definition gibt es nicht. Die meisten Institutionen definieren den Beginn des Weltraums in einer Höhe von 100 km über dem Meeresspiegel, an der sogenannten Kármán Linie \cite[vgl. 2.b)]{fai_weltraum}. Jenseits dieser Höhe ist die Zentrifugalkraft größer als die aerodynamischen Kräfte, daher wird ab 100 km nicht mehr von Luftfahrt, sondern von Raumfahrt gesprochen.





\chapter{Turm oder Aufzug}



\chapter{Aufzug}

\section{Seil}



\section{Design}

Nachdem gezeigt worden ist, dass ein Seil ins Weltall stabil gespannt werden kann, widmen wir uns drei entscheidenden Designfragen: 
\begin{itemize}
\item Welches Material benötigen wir für das Seil?
\item Wie können wir Stabilität und Kosteneinsparungen vereinen?
\item Wie wird das Seil angebracht?
\end{itemize}

\section{Materialfrage}
\textsl{Geschrieben von Veit Simoneit}\\
Der limitierende Faktor beim Bau eines Aufzugs oder eines Turms bis zum geostationären Orbit ist das verwendete Material. Das Gewicht des Materials ist ein wichtiger Faktor, jedoch noch elementarer ist die spezifische Höhe des Materials. Spezifische Höhe, auch ''self support length'' genannt ist die maximale Länge eines Pfeilers aus einem spezfischen Material, der nur an der Spitze befestigt ist und der sein Eigengewicht tragen kann.\cite{wiki:Specific_strength}

Ohne Beschränkung der Allgemeinheit nehmen wir für die spezifische Höhe überall am Seil die gleiche Gravitation von $g_0 = 9,80665 m/s^2$ an.
Betrachten wir im Vergleich drei Materialien: Stahl, Kevlar und Carbon Nano Tubes. \cite[vergleiche]{ED00}
\subsection{Stahl}
Das klassische Baumaterial um Wolkenkratzer überall auf der Welt zu bauen: Stahl. Da Stahl per se kein homogenes Material sein muss, betrachten wir die physikalischen Werte für stainless steel ohne die allgemeine Gültigkeit zu verlieren. Die Dichte von Stahl wird angegeben mit $\varrho_S = 7850 kg/m^3$ \cite[Vgl.]{PE75}. Die Zugfestigkeit von Stahl beträgt $2*10^9 kg/m*s^2$ , was $2GPa$ entspricht. Man berechnet die spezifische Höhe nun durch die Formel 
\begin{equation}
\frac{Zugfestigkeit_{Stahl}}{Dichte_{Stahl}*g_0} = \frac{2*10^9}{7850*9,80665}
<=> 25 980 m \sim 26 km 
\end{equation}

Überprüfen wir kurz die Einheiten
\begin{equation}
\frac{\frac{kg}{m*s^2}}{\frac{kg}{m^3}*\frac{m^2}{s^2}}= \frac{\frac{kg}{m*s^2}}{\frac{kg*m^2}{m^3*s^2}} = \frac{1}{\frac{1}{m}} = m
\end{equation}
Bei einer Seildicke von $2 cm$ im Durchmesser und einer Länge von $144.000 km$ bedeutet dies eine Gesamtmasse von $\sim 355.000 t$.\cite{PE75}
Betrachten wir ein anderes Vergleichsmaterial.
\subsection{Kevlar}
Kevlar wird in modernen schusssicheren Westen verwendet, da es leichter als Stahl aber sehr stabil ist. Die Dichte von Kevlar wird mit $\varrho_K = 1440 kg/m^3$ angegeben und liegt somit bei ungefähr einem Sechstel der Dichte von Stahl. Die Zugfestigkeit von Kevlar liegt bei $3,6 GPa$ und ist folglich 1,8-mal zugfester als Stahl. Hieraus folgt die spezifische Höhe für Kevlar mit
\begin{equation}
\frac{Zugfestigkeit_{Kevlar}}{Dichte_{Kevlar}*g_0} = \frac{3,6*10^9}{1440*9,80665}
<=> 254 929 m \sim 254 km 
\end{equation}
Kevlar hat eine spezifische Höhe, die 10-mal so hoch ist wie die von Stahl, jedoch ist es bis jetzt nicht ausreichend um einen Space Elevator zu bauen. \cite{PE75}
\subsection{Carbon Nano Tubes}
\label{sec:cnt}
Die Lösung der Materialfrage liegt in den \textsl{Carbon Nano Tubes}.
Bei einer theoretischen Zugfestigkeit von $30 - 130 GPa$ und einer Dichte von $\varrho_{CNT} = 1300 kg/m^3$ liegt die spezifische Höhe zwischen $2.350 - 10.000 km$. Laut Pearson ist der Bau eines Weltraumaufzugs, bzw. Turms vergleichbar mit dem Bau eines Turms von $4900 km$ in einem gleichmäßigen 1-g-Schwerefeld \cite{PE75}. Und genau diese Voraussetzung erfüllen die Carbon Nano Tubes.
\section{Tapering}
\subsection{Tapering-Modellierung}
Es bleiben zwei Probleme bei unseren Materialanfordungen an unser Seil, die mit unserem bisherigen Modellansatz nur schwer in den Griff zu bekommen sind.

Zum einen sind Carbon Nano Tubes teuer. Selbst vergleichsweise billige Proben sind ab \$60 pro Gram erhältlich. Für Material von hoher Qualität, wie sie für diesen Anwendungszweck erforderlich wären, fallen sogar noch extremere Kosten von \$750 pro Gramm\cite{baughman2002carbon}.

Sogar wesentlich gravierender ist die Gefahr, dass die in Abschnitt \ref{sec:cnt} genannte Spitzenlast von 130 GPa gar nicht zu erreichen sind. Es gibt Gründe anzunehmen, dass Carbon Nano Tubes einer theoretischen Belastbarkeitsgrenze von 45 GPa unterliegen\cite{pugno2007space}.

Beide Bedenken liefern uns gute Gründe so sparsam wie möglich mit unserem Kabelmaterial umzugehen. Da wir bestimmte Mindestanforderungen an die Länge unseres Aufzugs haben, nämlich dass er auf einer Seite bis zur Erdoberfläche reichen sollte und andererseits mindestens bis über den geosynchronen Orbit hinaus, bleibt uns eigentlich nur die Option an der Fläche zu sparen.\\
Im Falle der Kosten ist offensichtlich, dass eine Reduktion der Gesamtmenge des Materials zu einer Kostenersparnis führt.\\
Im Falle der Belastbarkeit ist dies weniger offensichtlich, da sowohl die maximal mögliche Belastung des Seils (siehe Gleichung \ref{eq:Festigkeit}), wie auch die auf das Seil wirkenden Kräfte (siehe Gleichung \ref{eq:kraft}), linear mit der Fläche skalieren. Der Trick ist, dass unser bisheriges Modell von einer konstanten Querschnittfläche A des Seils ausgeht. Ziel bei der Materialfindung war es demnach ein Material mit Zugfestigkeit $\sigma$ und Dichte $\varrho$ zu finden, so dass die Ungleichung \ref{eq:notaper} erfüllt ist.

\begin{equation}
A \cdot \sigma \geq \int_{R_E}^{r} A \cdot \varrho \cdot g_0 \cdot M_E \cdot \frac{1}{\tilde{r}^2} - A \cdot \varrho \cdot \omega \cdot \tilde{r}\ d\tilde{r}
\label{eq:notaper}
\end{equation}

Hierbei ist $A$ die Fläche des Seils, $M_E$ ist die Masse der Erde, $R_E$ der Radius derselbigen, $\omega$ ist die Winkelgeschwindigkeit.

Beachtet man jedoch, dass an unterschiedlichen Stellen im Seil eine unterschiedlich hohe Zugkraft herrscht, so kann man die Materialmenge der entsprechenden Zugkraft anpassen.
Wir variieren A nach der Höhe r und setzen $Z(r) = \lambda \cdot A(r)$ und erhalten die Integralgleichung \ref{eq:taperint}.

\begin{equation}
\lambda A(r) = \int_{R_E}^{r} A(r) \cdot \varrho \cdot g_0 \cdot M_E \cdot \frac{1}{\tilde{r}^2} - A(r) \cdot \varrho \cdot \omega \cdot \tilde{r}\ d\tilde{r}
\label{eq:taperint}
\end{equation}

%TODO justification

Der Vorteil dieser Modellierung gegenüber der Formulierung des Problems mit konstanter Querschnittfläche ist, dass sofern unabhängig von den vorkommenden Konstanten Lösungen für diese Integralgleichung existieren, sich ein stabiles Seil für jedes erdenkliche Material konstruieren lässt, wenn auch nur theoretisch. Es sollte offensichtlich sein, dass sich aus der bloßen theoretischen Existenz dieser Lösung noch keine praktische Umsetzbarkeit ergibt.

\subsection{Lösung für Tapering-Modell}

Um die Lösungen der Integralgleichung zu bestimmen, wenden wir zunächst den Fundamentalsatz der Analysis an und formen die Gleichung so zur Differentialgleichung \ref{eq:taperdgl} um.
\begin{equation}
A'(r) \cdot \sigma = A(r) \cdot \varrho \cdot g_0 \cdot M_E \cdot \frac{1}{\tilde{r}^2} - A(r) \cdot \varrho \cdot \omega \cdot \tilde{r}\ d\tilde{r}
\label{eq:taperdgl}
\end{equation}

Dies ist eine nicht autonome gewöhnliche homogene lineare Differentialgleichung 1. Ordnung.

%TODO justify

In der Literatur\cite{PE75} findet man sie auch in der Form \ref{eq:taperdgllit}.

\begin{equation}
A'(r) \cdot \sigma = \varrho \cdot g_E \cdot {r_E}^2 \cdot (\frac{1}{r^2} - \frac{r}{{r_{geo}}^3}) \cdot A(r)\ dr
\label{eq:taperdgllit}
\end{equation}

%TODO justify

Da unsere Differentialgleichung \ref{eq:taperdgl} separierbar ist, können wir sie sehr einfach durch Trennung der Variablen lösen.

%TODO do it

Wir erhalten nun also eine Lösung der Form \ref{eq:tapersolution}

\begin{align}
A(r) &= c_I * e^{\frac{-2\rho \cdot m_E \cdot G - \rho \omega^2 \cdot r^3}{2 r \sigma}}\label{eq:tapersolution}\\
c_I &= A(r_0) \cdot e^{-\frac{-2 c \rho m_E G - \rho \omega^2 r_0^3}{2 r_0 \sigma}}
\end{align}

Analog zur obigen Umformung \ref{eq:taperdgllit} unserer ursprünglichen Differentialgleichung \ref{eq:taperdgl} findet man in der Literatur auch die Lösung \ref{eq:tapersolutionlit}

\begin{equation}
A(r)=A_{geo}*e^\frac{3r_E^2}{2*h*r_{geo}}*e^{\frac{-r_E}{h}*(\frac{r_E}{r}+\frac{r_E*r^2}{2r_{geo}^3})}
\label{eq:tapersolutionlit}
\end{equation}

Basierend auf dieser Lösung lässt sich das so genannte Taperverhältnis bestimmen. Dies ist der Quotient zwischen der Querschnittfläche auf der Erdoberfläche und der 

\begin{equation}
\frac{A}{A_0} = e^\frac{0,776*r_E}{h}
\label{eq:taperratio}
\end{equation}
$h=\frac{\sigma}{\varrho*g_0}$ mit $\sigma$ ist die als konstant(!) gesetzte Belastung. Man erkennt mit zunehmendem $h$ nimmt das Taperverhältnis ab.
Aus Gleichung \ref{eq:taperdgl} folgt, dass die größte Zugkraft am geostationären Orbit auf das Seil wirkt. Dies kann auch daraus geschlussfolgert werden, dass an diesem Punkt zwei gleich große Kräfte in entgegengesetzte Richtungen wirken. Zum einen die Kraft des Seils, das Richtung Erde zieht und zum anderen der Teil des Seils, der vom geostationären Orbit aus von der Erde weg zeigt. Man beachte, dass der geostationäre Orbit genau der Punkt ist, an dem sich die Gravitationskraft der Erde und die Fliehkraft, die auf das Objekt wirkt gegenseitig aufheben, wäre eine Kraft größer, so würde sich das Objekt auf eine andere Umlaufbahn bewegen. Daraus folgt auch, dass die Kraft, die auf den geostationären Punkt wirkt, maximal ist. \\
In Abbildung \ref{fig:Tapering} erkennt man, wie das Taperingverhältnis zunimmt bis zum geostationären Orbit und dann wieder abnimmt. Die Kurve ist nicht symmetrisch, da die Gravitation von der Erde weg mit einem anderen Faktor abnimmt, als sie zur Erde hin zunimmt.
\begin{figure}[!htb]
\centering
\includegraphics[scale=0.4]{Tapering.png} 
\caption{Kabel Taper Profil, entnommen aus Edwards(2000)\cite{ED00}}
\label{fig:Tapering}
\end{figure}
\\
Vergleicht man die drei Materialien aus dem vorherigen Abschnitt bezüglich ihres Taper Verhältnisses am geostationären Punkt, so erhält man die folgenden Werte:

\begin{itemize}
\item Stahl: \ \ \ \ $\frac{A_{geo}}{A_0} = 5,8 *10^82$
\item Kevlar:\ \ \   $\frac{A_{geo}}{A_0} = 2,71 *10^8$
\item CNT: \ \ \ \ $\frac{A_{geo}}{A_0} = 1,62\ bis\ 8,2 $ 
\end{itemize}

Auf Grund der Größenordnung des Verhältnisses ist die einzige Lösung für die Materialwahl Carbon Nano Tubes, Stahl und Kevlar und andere Materialien haben ein zu großes Taperverhältnis.

\subsection{Modell mit Nutzlast}

%TODO Nutzlast Modell

%TODO Nutzlast Rechnung

\section{Deployment}



\bibliography{mybib}{}
\bibliographystyle{plain}

\appendix
\chapter{xxx}
\section{xxx}



\end{document}